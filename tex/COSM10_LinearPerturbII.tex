\documentclass[CJK,13pt]{beamer}
\input{macros.tex}
\input{titlepage.tex}
  \date{}
  \begin{document}
  \bch
  \tpage{10}{Linear Perturbation Theory II}

  \secpage{捡起我们的小目标}{\addfig{1.1}{xiaomubiao.jpg}}
  

  \begin{frame}
    \frametitle{目前的进度}
    \bitem
  \item[\checkmark]{给出度规微扰的数学描述(难度系数 $\star\star$)}
  \item[\checkmark]{计算爱因斯坦张量(难度系数 $\star\star\star\star$)}
  \item[\checkmark]{计算各种成分的能量动量张量(难度系数 $\star\star\star$)}    
  \item[4]{写出CDM的运动方程(难度系数 $\star$)}
  \item[5]{写出baryon的运动方程(难度系数 $\star\star$)}        
  \item[6]{写出massless neutrino的玻尔兹曼方程(难度系数 $\star\star\star\star\star$)}
  \item[7]{写出photon的玻尔兹曼方程(难度系数 $\star\star\star\star\star\star\star\star\star\star$)}
  \item[8]{把所有方程写成代码(难度系数$\star\star\star\star\star\star\star\star$)}
    \eitem

  \end{frame}
  

  \section{CDM}

  \secpage{微目标4}{写出CDM运动方程}
  
    \begin{frame}
      \frametitle{暗物质密度扰动 $\delta_c$ 的演化方程}
      注意曲率势使得单位共动体积对应的物理体积变成了 $a^3(1-3\Psi)$,欧拉连续性方程成为:
      $$\frac{d\left(\bar{\rho}_c(1+\delta_c)(1-3\Psi)a^3\right)}{d\tau} + \nabla\cdot\left[\bar{\rho}_c(1+\delta_c)(1-3\Psi)a^3 \vec{u}_c\right] = 0.$$
      注意 $\bar{\rho}_c a^3$ 是常数,以及 $\vec{u}_c = \frac{d\vec{x}}{d\tau}$ 是小量,上式保留一阶近似为
      $$ \frac{d\delta_c}{d\tau} - 3\frac{d\Psi}{d\tau} + \nabla\cdot \vec{u}_c = 0 $$
      转化到傅立叶空间,并利用定义 $\upsilon_c\equiv i\frac{\vec{k}}{k}\cdot\vec{u}_c$,即得到
      {\blue      $$ \frac{d\delta_c}{d\tau} = 3\frac{d\Psi}{d\tau} - k\upsilon_c $$}
      和第7讲忽略度规扰动的方程相比,方程右边的 $3\frac{d\Psi}{d\tau}$ 是在牛顿规范下的相对论修正。在较小尺度上,由于要考虑的 $k$ 都很大,可以忽略这个修正。
    \end{frame}
    

    \begin{frame}
      \frametitle{暗物质速度归一化散度 $\upsilon_c$ 的演化方程}
      第7讲的冷物质速度演化方程
      $$\frac{d\left[a^2\frac{d\vec{x}}{dt}\right]}{dt} = -\nabla\Phi$$
      可以直接使用(因为已经包含度规扰动 $\Phi$ 了。作替换 $d\tau = \frac{dt}{a}$ 之后,方程是:
        $$\frac{1}{a}\frac{d\left[a\vec{u}_c\right]}{d\tau} = -\nabla\Phi$$
        两边左点乘 $\nabla$ 得到
        $$\frac{1}{a}\frac{d(a\nabla\cdot \vec{u}_c)}{d\tau} = -\nabla^2\Phi $$
        最后转化到傅立叶空间
        {\blue $$ \frac{d\upsilon_c}{d\tau} = -aH\upsilon_c  + k \Phi $$}       
    \end{frame}

  
  \begin{frame}
    \frametitle{目前的进度}
    \bitem
  \item[\checkmark]{给出度规微扰的数学描述(难度系数 $\star\star$)}
  \item[\checkmark]{计算爱因斯坦张量(难度系数 $\star\star\star\star$)}
  \item[\checkmark]{计算各种成分的能量动量张量(难度系数 $\star\star\star$)}    
  \item[\checkmark]{写出CDM的运动方程(难度系数 $\star$)}
  \item[5]{写出baryon的运动方程(难度系数 $\star\star$)}        
  \item[6]{写出massless neutrino的玻尔兹曼方程(难度系数 $\star\star\star\star\star$)}
  \item[7]{写出photon的玻尔兹曼方程(难度系数 $\star\star\star\star\star\star\star\star\star\star$)}
  \item[8]{把所有方程写成代码(难度系数$\star\star\star\star\star\star\star\star$)}
    \eitem
  \end{frame}

  \section{Baryon}

  \secpage{微目标5}{写出baryon运动方程}
  
    \begin{frame}
      \frametitle{重子密度扰动 $\delta_b$ 的演化方程}
      虽然电子会和光子碰撞,但局域碰撞不影响描述物质守恒的欧拉方程。 如果忽略压强,重子的密度扰动和暗物质密度扰动满足一样的方程:
      {\blue      $$ \frac{d\delta_b}{d\tau} = 3\frac{d\Psi}{d\tau} - k\upsilon_b $$}
      在CAMB中,实际还引入了重子声速的计算,并对重子压强产生的效应做了修正。不过这是非常小的效应,在我们最简版本里暂不讨论。
    \end{frame}
    

    
    \begin{frame}
      \frametitle{重子速度归一化散度 $\upsilon_b$ 的演化方程}
      重子的动量演化要考虑到电子和光子的碰撞 (电子和原子核之间的库仑作用使得电子的动量可以有效传递给所有重子成分)。在相对论情形,光子流体元的动量密度为 $(\rho_\gamma+p_\gamma)\vec{u}_\gamma=\frac{4\rho_\gamma}{3}\vec{u}_\gamma$。 从重子参考系看,所有光子都发生单次Thomson散射时,传递给重子的净动量密度为
      $$\sum P_{\gamma\rightarrow b} = \frac{4}{3}\rho_{\gamma}\left(\vec{u}_\gamma-\vec{u}_b\right)$$     
      乘以每个光子被散射的平均次数 $n_e\sigma_Tcdt = an_e\sigma_T c d\tau$,再除以$\rho_b$转化为了流体加速度:
      $$\frac{d\vec{u}_b}{d\tau} = \ldots + \frac{4\rho_{\gamma}}{3\rho_b}an_e\sigma_Tc\left(\vec{u}_\gamma-\vec{u}_b\right) $$
      然后变换到傅立叶空间,取归一化散度把 $\vec{u}_b$ 转化为 $\upsilon_b$:
       $$ \frac{d\upsilon_b}{d\tau} = -aH\upsilon_b  + k \Phi + \frac{4\rho_{\gamma}}{3\rho_b} an_e\sigma_Tc \left(\upsilon_\gamma-\upsilon_b\right)$$
    \end{frame}


    \begin{frame}
      \frametitle{光子的 $\upsilon_\gamma$}
      我们已经讨论过光子的流体描述和分布函数微扰的关系:
      $$\upsilon_\gamma = \frac{1}{4}F_{\gamma 1}$$
      我们加上了下标 $\gamma$ 以区分光子和中微子的 $F_\ell$。


      另外, $\frac{\rho_\gamma }{\rho_b} a = \frac{\Omega_{\gamma}}{\Omega_b}$ 是常数(注意采用的是 $a_0=1$ 的归一化习惯), 
      所以最后我们会使用的方程是:
      {\blue $$ \frac{d\upsilon_b}{d\tau} = -aH\upsilon_b  + k \Phi + \frac{4\Omega_{\gamma}}{3\Omega_b} n_e\sigma_Tc \left(\frac{1}{4}F_{\gamma 1}-\upsilon_b\right)$$}
    \end{frame}


    \begin{frame}
      \frametitle{和 RECFAST 对接}
      自由电子数密度 $n_e(z)$ 可以用
      $$ n_e(z) = n_HX_e(z) = \frac{\frac{3H_0^2}{8\pi G}\Omega_b(1-Y_P)}{m_H}(1+z)^3 X_e(z) $$
      以及RECFAST输出的 $X_e(z)$ 算出来。
    \end{frame}
    
  \begin{frame}
    \frametitle{目前的进度}
    \bitem
  \item[\checkmark]{给出度规微扰的数学描述(难度系数 $\star\star$)}
  \item[\checkmark]{计算爱因斯坦张量(难度系数 $\star\star\star\star$)}
  \item[\checkmark]{计算各种成分的能量动量张量(难度系数 $\star\star\star$)}    
  \item[\checkmark]{写出CDM的运动方程(难度系数 $\star$)}
  \item[\checkmark]{写出baryon的运动方程(难度系数 $\star\star$)}        
  \item[6]{写出massless neutrino的玻尔兹曼方程(难度系数 $\star\star\star\star\star$)}
  \item[7]{写出photon的玻尔兹曼方程(难度系数 $\star\star\star\star\star\star\star\star\star\star$)}
  \item[8]{把所有方程写成代码(难度系数$\star\star\star\star\star\star\star\star$)}
    \eitem
  \end{frame}


    \section{massless neutrinos}

    \secpage{微目标6}{中微子玻尔兹曼方程}
    
    \begin{frame}
      \frametitle{中微子的玻尔兹曼方程}
      碰撞可以忽略的中微子要用完整的玻尔兹曼方程描述 $f(\tau, \vec{x}, \vec{q})$ 的演化:
      $$\frac{\partial f}{\partial \tau} + \frac{dx^i}{d\tau}\frac{\partial f}{\partial x^i} + \frac{d q}{d\tau} \frac{d f_0}{d q} = 0$$
      注意最后一项里我们只保留了 $\frac{d f_0}{d q}$ 的贡献,而未考虑完整的 $\delta f$ 对 $\vec{q}$ 的偏导。这是因为 $\frac{d\vec{q}}{d\tau}$ 已经是一阶小量,和它相乘的因子里只要保留零阶项就行了。

      同样的理由,由于 $\frac{\partial f}{\partial x^i}$ 是一阶小量,我们只要保留 $\frac{dx^i}{d\tau}$ 的零阶近似 $\frac{d\vec{x}}{d\tau} = a \frac{d\vec{x}}{dt} \approx \frac{\vec{q}/a}{\sqrt{(q/a)^2+m^2}}$。忽略中微子质量时, $\frac{d\vec{x}}{d\tau}\approx \hat{q}$。

      \notes{注意玻尔兹曼方程里是单粒子的 $\frac{dx^i}{d\tau}$,和之前的理想流体近似里的流体元(多粒子平均)速度不一样。}
    \end{frame}

    \begin{frame}
      \frametitle{中微子的玻尔兹曼方程(续)}
      变换到傅立叶空间,利用 $\hat{q}\cdot\vec{k} = k\mu $,有
      \begin{equation}
        \frac{\partial \delta f}{\partial \tau} + ik\mu \delta f + \frac{d q}{d\tau} \frac{d f_0}{d q} = 0. \label{eq:nu1}
      \end{equation}
      附录中给出了 $\frac{dq}{d\tau}$ 的计算。 把结果 $\frac{dq}{d\tau} = q\left(\Psi'-ik\mu\Phi\right)$ 代入 \eqref{eq:nu1} 后就得到
      $$\frac{\partial \delta f}{\partial \tau} + ik\mu \delta f +  q\left(\Psi'-ik\mu\Phi\right) \frac{d f_0}{d q} = 0.$$      
      两边对 $q^3dq$ 积分,并除以  $\int f_0(q)q^3dq$, 得到 $F(\tau, k, \mu)$ 的演化方程:
      $$\frac{\partial F}{\partial \tau} + ik\mu F + \left(\Psi'-ik\mu\Phi\right) \frac{\int q^4\frac{d f_0}{d q}dq}{\int q^3f_0dq} = 0.$$      
    \end{frame}


    \begin{frame}
      \frametitle{中微子的玻尔兹曼方程(续)}
      利用 $f_0(q) \propto \frac{1}{e^{\frac{q}{T_{CNB}}}+1}$,直接计算最后一项的积分,得到
       $$\frac{\partial F}{\partial \tau} + ik\mu F + 4\left(ik\mu\Phi-\Psi'\right) = 0.$$
      代入Legendre多项式展开 $ F(\tau,k,\mu) =  \sum_{\ell=0}^\infty (-i)^\ell F_\ell(\tau, k) P_\ell(\mu) $,并利用递推公式
      $\mu P_\ell(\mu) = \frac{\ell+1}{2\ell+1}P_{\ell +1}(\mu) + \frac{\ell}{2\ell+1}P_{\ell-1}(\mu)$,得到:
      $$ \left[\frac{\partial F_\ell}{\partial \tau} - k\frac{\ell}{2\ell-1} F_{\ell-1} + k \frac{\ell+1}{2\ell+3} F_{\ell+1}\right]   -  4 k\Phi \delta_{\ell 1} -4\Psi'\delta_{\ell 0} = 0.$$

    \end{frame}


    \begin{frame}
      \frametitle{中微子的玻尔兹曼方程(续)}
      对固定的 $k$ 而言,$F_\ell$ 就只是 $\tau$ 的函数,所以我们把 $\frac{\partial F_\ell}{\partial \tau}$ 换成 $\frac{dF_\ell}{d\tau}$。另外,我们把所有下标都标注上 $\nu$ 以表示这是中微子的分布函数扰动。      

      实际的数值计算中,我们不可能无限地算无穷多的 $F_\ell$,所以我们将采用截断 $\ell_{\max}$。这样 $F_{\nu \ell_{\max}}$ 的演化方程就不封闭了。我们采用一种被实践证明是不错的近似:
      $$\frac{d F_{\nu \ell_{\max}}}{d\tau} = k \frac{2\ell_{\max}+1}{2\ell_{\max}-1} F_{\nu,\ell_{\max}-1}-\frac{\ell_{\max}+1}{\tau}F_{\nu\ell_{\max}}$$
      这个近似只是为了提高精度而采用的。它当然并不关键,只要你 $\ell_{\max}$ 取得够大,直接就随便用个 $F_{\nu\ell_{\max}}=0$ 也不会有什么问题(当然,代价是代码运行速度变慢了)。
    \end{frame}

    \begin{frame}
      \frametitle{中微子(忽略质量)的演化方程}
      最后写到一起,就是
     {\blue  \bea
      \frac{d F_{\nu 0}}{d \tau} &=& -\frac{1}{3}kF_{\nu 1} + 4\Psi' \newl
      \frac{d F_{\nu 1}}{d \tau} &=& k\left(F_{\nu 0}-\frac{2}{5}F_{\nu 2} + 4\Phi\right) \newl
      \frac{d F_{\nu \ell}}{d\tau} &=& k\left[\frac{\ell}{2\ell-1} F_{\nu,\ell-1}-\frac{\ell+1}{2\ell+3}F_{\nu, \ell+1}\right], \   2\le \ell < \ell_{\max} \newl
      \frac{d F_{\nu \ell_{\max}}}{d\tau} &=& k \frac{2\ell_{\max}+1}{2\ell_{\max}-1} F_{\nu,\ell_{\max}-1}-\frac{\ell_{\max}+1}{\tau}F_{\nu\ell_{\max}}      
      \eea
      }
    \end{frame}


    
  \begin{frame}
    \frametitle{目前的进度}
    \bitem
  \item[\checkmark]{给出度规微扰的数学描述(难度系数 $\star\star$)}
  \item[\checkmark]{计算爱因斯坦张量(难度系数 $\star\star\star\star$)}
  \item[\checkmark]{计算各种成分的能量动量张量(难度系数 $\star\star\star$)}    
  \item[\checkmark]{写出CDM的运动方程(难度系数 $\star$)}
  \item[\checkmark]{写出baryon的运动方程(难度系数 $\star\star$)}        
  \item[\checkmark]{写出massless neutrino的玻尔兹曼方程(难度系数 $\star\star\star\star\star$)}
  \item[7]{写出photon的玻尔兹曼方程(难度系数 $\star\star\star\star\star\star\star\star\star\star$)}
  \item[8]{把所有方程写成代码(难度系数$\star\star\star\star\star\star\star\star$)}
    \eitem

    \skipline
    
    再坚持亿下,\sout{反正永远都不可能} 就快要完成小目标了。
  \end{frame}
    

  \section{Appendix}
  \secpage{附录}{ $\frac{dq}{d\tau}$ 的计算}

  \begin{frame}
    \frametitle{ $\frac{dq}{d\tau}$ 的计算}
    我们使用测地线方程的另一种等价形式来推导 $\frac{dq}{d\tau}$:
    {\blue    $$ \frac{d p_\lambda}{d x^0} = \frac{1}{2}g_{\mu\nu,\lambda}\frac{p^\mu p^\nu}{p^0}$$}
    注意到上一讲中给出的 $p_i = -(1-\Psi)q_i$,$p^i = \frac{1}{a^2}(1+\Psi)q^i$, 以及 $p^0 = \frac{1-\Phi}{a^2}\sqrt{q^2+m^2a^2}$,测地线方程保留到一阶小量,就有
    $$-\frac{dq_i}{d\tau} + q_i\frac{d\Psi}{d\tau}  = \partial_i\Phi \sqrt{q^2+m^2a^2} +  \frac{q^2}{\sqrt{q^2+m^2a^2}} \partial_i\Psi$$
  \end{frame}

  \begin{frame}
    \frametitle{ $\frac{dq}{d\tau}$ 的计算}    
    注意到粒子在移动,所以 $\frac{d\Psi}{d\tau}$ 和我们约定的 $\Psi' \equiv \frac{\partial\Psi}{\partial\tau}$ 不同,
    $$\frac{d\Psi}{d\tau} = \Psi' + \partial_j\Psi \frac{dx^j}{d\tau} = \Psi' +  \frac{q^j}{\sqrt{q^2+m^2a^2}}\partial_j\Psi.$$
    于是就有
   {\small $$-\frac{dq_i}{d\tau} + q_i\left(\Psi' +  \frac{q^j}{\sqrt{q^2+m^2a^2}} \partial_j\Psi \right) = \partial_i\Phi \sqrt{q^2+m^2a^2} +  \frac{q^2}{\sqrt{q^2+m^2a^2}}\partial_i\Psi $$}
  \end{frame}


  \begin{frame}
    \frametitle{ $\frac{dq}{d\tau}$ 的计算}
    写到傅立叶空间,
   {\small $$-\frac{dq_i}{d\tau} + q_i\left(\Psi' + \ii   \frac{kq\mu}{\sqrt{q^2+m^2a^2}} \Psi \right)= \ii k_i \Phi \sqrt{q^2+m^2a^2} + \ii k_i\frac{q^2}{\sqrt{q^2+m^2a^2}}\Psi  $$}
    两边同乘以 $q^i$ 并对 $i$ 求和,
    $$-q\frac{dq}{d\tau} + \Psi' q^2  = \ii kq \mu\sqrt{q^2+m^2a^2}\Phi.$$
    即
    $$\frac{dq}{d\tau} = q\Psi' - \ii k\mu\sqrt{q^2+m^2a^2} \Phi$$
    
  \end{frame}
  
  \ech
\end{document}




  
