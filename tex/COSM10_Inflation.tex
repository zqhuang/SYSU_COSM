\documentclass[CJK,13pt]{beamer}
\input{macros.tex}
\input{titlepage.tex}
  \date{}
  \begin{document}
  \bch
  \tpage{7}{Inflation}

  \section{Motivation of Inflation}
  \secpage{Motivation of Inflation}{Flatness Problem and Horizon Problem}

  \thinka{假设我们的宇宙是 $\Omega_m=0.3$ 的 $\Lambda$CDM 宇宙,在宇宙学红移为 $10^8$ 时有超级智能生命发现用当时的 $\Omega_k=-\frac{k}{a^2H^2}\vert_{z=10^8}=10^{-3}$ (注意这里我们滥用了一下$\Omega_k$符号),那么我们目前的 $\Omega_k=-\frac{k}{a_0^2H_0^2}$ 应该是多大?}
  
  \begin{frame}
    \frametitle{Flatness Problem}
    除了宇宙学常数面临coincidence problem,空间曲率也是如此:如果早期宇宙的物质动能和引力势能不严格抵消, 那么我们目前应该看到一个非常不平坦 ($\Omega_k$ 偏离零非常明显) 的宇宙。目前CMB观测支持的 $\Omega_k\approx 0$ 令人困惑:是什么机制让早期宇宙的物质动能和引力势能抵消得如此完美?

    \addfig{2}{tangping.jpg}
    
  \end{frame}


  \begin{frame}
    \frametitle{哈勃视界}
    \addfig{2}{zuiyuan.jpg}

    假设我们的宇宙是 $\Omega_m=0.3$ 的 平坦$\Lambda$CDM 宇宙。有一种神奇的永生生命叫蓝精灵。我们看到在宇宙学红移为 $1000$ 时,蓝精灵Bob和蓝精灵Alice的所在的天空方位角相差一度,并且Bob向Alice发了一封表白信。假设表白信以光速运动。问:现在Alice收到Bob的信了吗?
    
  \end{frame}
  
  \begin{frame}
    \frametitle{Horizon Problem}
    
    
  \end{frame}
  
    \ech
\end{document}




  
