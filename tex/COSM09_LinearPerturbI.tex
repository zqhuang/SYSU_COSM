\documentclass[CJK,13pt]{beamer}
\input{macros.tex}
\input{titlepage.tex}
  \date{}
  \begin{document}
  \bch
  \tpage{9}{Linear Perturbation Theory I}

  \section{Plan}

  \secpage{小目标}{\addfig{1.1}{xiaomubiao.jpg}}
  
  \begin{frame}

    \frametitle{颇具野心的计划}
    我们前面已经介绍了如何用结构增长方程描述“冷物质密度扰动” 。但实际上宇宙中除了冷物质之外,至少还有光子、中微子这些成分存在非均匀性。另外,时空度规的非均匀性具有6个物理自由度,并不能简单地用一个牛顿引力势来描述。


    \skipline

    所以,一个颇具野心的计划是建立一套完整的方程组,描述所有成分以及时空度规的非均匀性(宇宙学微扰)。

    \skipline
    
    但是作为一名\sout{毕不了业的}对现实世界有深刻认识的 研究僧,你肯定猜得到这件事已经有人做了——
  \end{frame}

  \begin{frame}
    \frametitle{完成了上述计划的代码: CAMB 和 CLASS}
    CAMB 和 CLASS 是两个与时俱进的、求解宇宙学微扰的代码。

    \skipline

    所谓与时俱进,就是为了追求速度、精度以及通用性,这两个代码都被打造得极为精致复杂,以至于 ——

    \addfig{0.9}{haogaoji.jpg}

    {\bf\large 你不太可能看得懂。}
  \end{frame}


  \begin{frame}
    \frametitle{代码到底复杂在哪里?}
    \bitem
  \item{在欧式空间通常用Fourier变换来研究线性系统;但当 $\Omega_k\ne 0$ 时,要改为 \&¥\#*\%\# 变换 (此处省略一万行特殊函数的定义以及数值算法)。}
  \item{度规的非均匀性很复杂,有由于可以任意人为选取坐标系带来的4个“规范自由度”,还有牛顿引力势(1个自由度)、曲率势(1个自由度)、无源矢量场扰动(2个自由度)、引力波(2个自由度)。}
  \item{光子和重子在recombination之前处于强耦合状态,描述它们演化的方程数值上非常不稳(其实是秒崩)。因此需要换一个近似但是稳定的算法。为了提高计算精度,近似算法被搞得有亿点点复杂。}    
  \item{光子在晚期宇宙占的比重较低,因此可以采用各种近似手段来提升计算速度。}
  \item{中微子在晚期宇宙变得非相对论性,动力学相对较复杂。}
    \eitem
  \end{frame}


  \begin{frame}
    \frametitle{小目标}
    我们的小目标是写一个看得懂的代码来计算宇宙学微扰。为此,我们需要把上述各种复杂情形尽可能地简化。具体方案如下:

    \addfig{0.8}{xiaomubiao.jpg}
    
    \bitem
  \item{令 $\Omega_k=0$(解决不了问题,就解决提出问题的…)}
  \item{只考虑标量度规扰动(虽然至少还有引力波)}    
  \item{把中微子当成无质量粒子(虽然它们有…)}    
  \item{放弃对精度和计算速度的极致追求。}
    \eitem

    \notes{引力波的演化和中微子质量的影响其实都是重要的问题,在 CAMB 和 CLASS 里都会被认真地处理。不过我们在“最简版本”里将忽略这些。}
    
  \end{frame}


  
  \begin{frame}
    \frametitle{小目标的checklist (8个微目标)}
    \bitem
  \item[1]{给出度规微扰的数学描述(难度系数 $\star\star$)}
  \item[2]{计算爱因斯坦张量(难度系数 $\star\star\star\star$)}
  \item[3]{计算各种成分的能量动量张量(难度系数 $\star\star\star$)}    
  \item[4]{写出CDM的运动方程(难度系数 $\star$)}
  \item[5]{写出baryon的运动方程(难度系数 $\star\star$)}        
  \item[6]{写出massless neutrino的玻尔兹曼方程(难度系数 $\star\star\star\star\star$)}
  \item[7]{写出photon的玻尔兹曼方程(难度系数 $\star\star\star\star\star\star\star\star\star\star$)}
  \item[8]{把所有方程写成代码(难度系数$\star\star\star\star\star\star\star\star$)}
    \eitem
  \end{frame}


 \begin{frame}
   \frametitle{宇宙学微扰论里的Fourier变换}
    一阶微扰论所有方程是线性的,可以用Fourier变换把偏微分方程转化为常微分方程组。因此在接下来的讨论里,所有依赖于 $\vec{x} \equiv (x^1,x^2,x^3)$ 的量都会被傅立叶变换到对偶的 $\vec{k}\equiv (k_1, k_2, k_3)$ 空间(傅立叶空间)。
    例如傅立叶空间中牛顿引力势的定义为:
    $$\Phi(\tau, \vec{k})\equiv \frac{1}{(2\pi)^3}\iiint \Phi(\tau, \vec{x}) e^{-i\vec{k}\cdot\vec{x}}d^3\vec{x}$$
    请不要介意方程左右两边都使用了 $\Phi$ 符号,你根据参量是 $\vec{k}$ 还是 $\vec{x}$ 就可以判断具体是在哪个空间的表述。

    \skipline

    Fourier逆变换为
    $$\Phi(\tau, \vec{x}) = \iiint \Phi(\tau, \vec{k}) e^{i\vec{k}\cdot\vec{x}}d^3\vec{k}$$
  \end{frame}


  \begin{frame}
    \frametitle{归一化的 $\hat{k}$}
    注意我们始终把 $\vec{k}$ 当成三维欧氏空间的矢量(也就是说我们并不区分 $k^i$ 和 $k_i$)。前面傅立叶变换公式中的内积 $\vec{k}\cdot\vec{x}$ 也就是简单的分量乘积之和。

    \skipline
    
    我们经常要使用长度归一化的波矢
    {\blue $$\hat{k}\equiv \frac{\vec{k}}{k}$$}
    这里的 $k = \lvert\vec{k}\rvert$ 是 $\vec{k}$ 的大小。

    \skiplines
    
    \notes{$k = \lvert\vec{k}\rvert$ 和 FRW 度规里的空间曲率参数 $k$ 符号相同。在本讲中,我们总是假设空间曲率为零,并不会混淆。但一般情况下,要根据上下文判断 $k$ 是什么意思。}
  \end{frame}

  
  \section{Metric}
  \secpage{微目标1}{度规扰动}
  
  \begin{frame}
      \frametitle{共形时间(conformal time) }
        为了和大多数文献和代码保持一致,在讨论宇宙学微扰时我们将采用共形时间 $\tau$,它的定义为:
        $$ \tau \equiv \int_0^t \frac{dt'}{a(t')}.$$
        老规矩,目前的共形时间记作 $\tau_0$。


        \skipline

        根据定义, $d\tau = \frac{dt}{a}$,也就是 $dt = a d\tau$,所以 FRW 度规可以写成
        $$ ds^2 = a^2\left[d\tau^2 - \frac{dr^2}{1-kr^2} - r^2\left(d\theta^2+\sin^2\theta\,d\phi^2\right)\right],$$
        这里我们重新把 $a$ 看成是 $\tau$ 的函数。

        \skipline

        {\blue 我们把对 $\tau$ 的导数用 $'$ 来表示,例如 $a' \equiv \frac{da}{d\tau}$。共形哈勃参量则定义为
        $\mathcal{H} \equiv \frac{a'}{a} = \dot a = aH$。}
  \end{frame}



    \begin{frame}
      \frametitle{$\Omega_k=0$ 情况的直角坐标写法}
      按照计划,我们只讨论 $\Omega_k=0$ 的情况。这时可以把共动坐标 $(r,\theta,\phi)$ 写成直角坐标 $ (x^1, x^2, x^3)$ 的形式:
      $$ ds^2 = a^2\left(d\tau^2-\delta_{ij}dx^i dx^j\right)$$
      这里的尺度因子 $a$ 只依赖于 $\tau$。按照宇宙学通常的习惯,拉丁字母 $i, j$ 只对空间指标 $1,2,3$ 求和。
  \end{frame}


   \begin{frame}
      \frametitle{($\Omega_k=0$情况) 度规的最一般微扰}
      度规的十个成分的最一般微扰可以写成:
      $$ ds^2 = a^2\left[(1+2\Phi)d\tau^2+ 2B_id\tau dx^i - \left(\delta_{ij}+H_{ij}\right)dx^i dx^j\right]$$
      这里 $H_{ij}=H_{ji}$ 对应 6 个自由度, $B_i$ 对应 3 个自由度,$\Phi$ 对应 1 个自由度——它们都是依赖于时空坐标 $(\tau, x^1,x^2,x^3)$ 的函数。

      \skipline

      然后我们{\blue 按照固定时间的三维空间切片里的张量类型}进行分类:
      \bitem
      \item{$\Phi$ 是个标量;}

        \item{$B_i$ 可以分解为标量 $C$ 和横向矢量 $D_i$(即满足 $\partial^iD_i=0$ 的无源矢量)的组合
          $$ B_i = \partial_i C +  D_i$$}
          \eitem

      \notes{这里的修饰语“横向”指做了傅立叶变换后, $D$ 矢量和波矢 $\veck$ 垂直。标量 $C$ 为一个自由度,横向矢量(因为加了 $\partial^iD_i=0$ 的限制)只有2个自由度,加起来还是3个自由度。}
      
  \end{frame}

   \begin{frame}
      \frametitle{($\Omega_k=0$情况) 度规的最一般微扰}
      \bitem
    \item{$H_{ij}$ 可以分解为两个标量 $\Psi$ 和 $E$,一个横向矢量 $A_i$,和一个横向无迹的二阶对称张量(引力波) $h_{ij}$。
      $$ H_{ij} = -2\Psi \delta_{ij} + \left(\partial_i\partial_j - \frac{1}{3}\delta_{ij}\right)E + (\partial_iA_j + \partial_j A_i) + h_{ij}$$
      这里横向矢量  $A_i$ 满足 $\partial^iA_i=0$,横向无迹的对称张量 $h_{ij}$ 满足 $\partial^ih_{ij} = 0$ 以及 $h^i_{\, i} = 0$。


      \skiplines

      \notes{注意这里的指标升降都是在固定时间的切片内(三维欧式空间)进行,也就是说指标在上在下其实是一样的。\\
       $h_{ij}$ 因为满足三个横向条件 $\partial^ih_{ij}=0$ 以及一个无迹条件 $h^i_{\, i}=0$,自由度是 $6-3-1=2$ 个,对应引力波的两种极化。}

    }
      \eitem
  \end{frame}
   

  
  \begin{frame}
    \frametitle{去掉规范自由度}
    我们可以在四维时空里面取四个很小的函数 $\epsilon^\mu(\tau, x^1, x^2, x^3)$ ($\mu=0,1,2,3$),然后定义:
    $$\tilde{\tau} \equiv \tau + \epsilon^0(\tau, x^1, x^2, x^3).$$
    $$\tilde{x^i} \equiv x^i + \epsilon^i(\tau, x^1, x^2, x^3).$$
    如果我们重新定义坐标系,在原先坐标为 $(\tau, x^1, x^2, x^3)$ 的物理点,都用新的坐标 $(\tilde{\tau}, \tilde{x^1}, \tilde{x^2}, \tilde{x^3})$ 来标记。相应的度规里就会多出由 $\epsilon^0$ (标量)和 $\epsilon^i$ (标量+横向矢量)产生的成分。所以度规里的两个标量和一个横向矢量——总计4个自由度是由坐标系选取产生的“规范自由度”(非物理自由度)。


    \notes{请再次注意我们这里讨论的标量、矢量都是指(固定时间的)三维空间内的张量类型——并且只是在一阶近似意义下的。也就是说,你无须弄清到底固定时间指的是固定 $\tau$ 还是固定 $\tilde{\tau}$。}
  \end{frame}


  \begin{frame}
    \frametitle{牛顿规范 (Newtonian Gauge)}
    按照计划,我们要取数学公式相对简单的“牛顿规范”(也有叫Poisson规范的)。具体来说,我们取合适的 $\epsilon^\mu$ 抵消掉两个标量 $C$、 $E$ 和一个横向矢量  $D$,得到
    {\small  $$ ds^2 = a^2\left\{\left(1+2\Phi\right)d\tau^2 - \left[\left(1-2\Psi\right)\delta_{ij}+ \partial_iA_j+\partial_jA_i + h_{ij}\right]dx^idx^j\right\}$$}
    至于  $\epsilon^\mu$ 具体该满足什么条件,推导并不复杂,也和接下去的计算毫无关系,就不详细给出了。

    \skipline
    
    \notes{一种更加流行(例如CAMB数值计算采用)的规范是同步规范(synchronous gauge):取合适的 $\epsilon^\mu$ 抵消掉两个标量 $\Phi$、$C$ 和一个横向矢量 $D$,得到
      $$ ds^2=dt^2 - (\delta_{ij}+H_{ij}) dx^idx^j .$$ 同步规范相当于在每个“固定的物理点”放上时钟来计时为 $t$,这样就还有如何选择“固定的物理点”的问题。在CAMB里采用的是取CDM流体元不动的物理点的方法(CDM comoving synchronous gauge)。}
    
  \end{frame}


  \begin{frame}
    \frametitle{忽略矢量和引力波成分}
    按照计划,在牛顿规范中忽略矢量成分和引力波成分
      {\blue $$ ds^2 = a^2\left[\left(1+2\Phi\right)d\tau^2 - \left(1-2\Psi\right)\delta_{ij}dx^idx^j\right]$$}
      这里的 $\Phi$ 对应牛顿引力势,$\Psi$ 叫做曲率势——它描述局域的空间曲率扰动。

      \skipline

      {\blue 微扰 $\Phi,\Psi$ 都是小量,我们在计算中,总是仅保留到它们的一阶。}另外,$\Phi$ 对应牛顿引力势是广义相对论里熟知的结论,这里就不给出推导了。

      \skipline
      
      \notes{如果你要专门研究如何测量原初引力波,就不能忽略 $h_{ij}$,这样其实也不会导致问题复杂多少。不过为了保证小目标尽可能精简,就不展开讨论了。}
  \end{frame}

  \begin{frame}
    \frametitle{小目标的checklist}
    \bitem
  \item[\checkmark]{给出度规微扰的数学描述(难度系数 $\star\star$)}
  \item[2]{计算爱因斯坦张量(难度系数 $\star\star\star\star$)}
  \item[3]{计算各种成分的能量动量张量(难度系数 $\star\star\star$)}    
  \item[4]{写出CDM的运动方程(难度系数 $\star$)}
  \item[5]{写出baryon的运动方程(难度系数 $\star\star$)}        
  \item[6]{写出massless neutrino的玻尔兹曼方程(难度系数 $\star\star\star\star\star$)}
  \item[7]{写出photon的玻尔兹曼方程(难度系数 $\star\star\star\star\star\star\star\star\star\star$)}
  \item[8]{把所有方程写成代码(难度系数$\star\star\star\star\star\star\star\star$)}
    \eitem
  \end{frame}

  
  \section{Einstein Tensor}
  \secpage{微目标2}{爱因斯坦张量}

  \begin{frame}
    \frametitle{代码}
    对牛顿规范的度规
      {\blue $$ ds^2 = a^2\left[\left(1+2\Phi\right)d\tau^2 - \left(1-2\Psi\right)\delta_{ij}dx^idx^j\right]$$}    
      用代码
    \url{http://zhiqihuang.top/cosm/codes/NewtonianGauge.py}
    可以计算出connection, Ricci scalar, 和 Einstein tensor(均保留到一阶小量)。

    \skipline

    结果如下(灰色为零阶项,黑色为一阶项;一撇表示对 $\tau$ 求偏导, $\mathcal{H}\equiv\frac{a'}{a}$, $\partial_i\equiv \frac{\partial}{\partial x^i}$, $\nabla^2\equiv \sum_{i=1}^3 \partial_i^2$):
  \end{frame}

  \begin{frame}
    \frametitle{联络}
     \begin{eqnarray}
\Gamma^{0}_{00} &=&{\gray \mathcal{H} + } \Phi',   \\
\Gamma^{0}_{i0} &=&  {\partial_i} \Phi,   \\
\Gamma^{i}_{00} &=&  {\partial_i} \Phi,   \\
\Gamma^{0}_{ii} &=& {\gray \mathcal{H} }- 2 \mathcal{H}(\Phi +\Psi) - \Psi'  ,\ \text{($i$ not summed)} \\
\Gamma^{i}_{i0} &=& {\gray \mathcal{H} }  -  \Psi' ,\ \text{($i$ not summed)} \\
\Gamma^{i}_{ji} &=& -  {\partial_j} \Psi ,\ \text{($i$ not summed)} \\
\Gamma^{j}_{ii} &=&  {\partial_j} \Psi  ,\ \text{($i$ not summed, and $i\ne j$)} 
\end{eqnarray}
  \end{frame}
  
  \begin{frame}
    \frametitle{Ricci 标量}
    \begin{eqnarray}
      R &=& {\gray - \frac{6 {a''}}{a^3 }} \nonumber \\
      && + \frac{12 \Phi  {a''}}{a^3 } + \frac{2 \nabla^2 \Phi }{a^2 } + \frac{6 \mathcal{H} \Phi' }{a^2 } \nonumber \\
      && + \frac{6\Psi'' }{a^2 } - \frac{4 \nabla^2 \Psi }{a^2 }  + \frac{18\mathcal{H}  \Psi'  }{a^2 }
      \end{eqnarray}
  \end{frame}

  \begin{frame}
    \frametitle{Einstein 张量}
 \begin{eqnarray}
G^{0}_{\ 0} &=& {\gray \frac{3\mathcal{H}^2}{a^2 }} - \frac{6 \mathcal{H}^2  \Phi }{a^2 } + \frac{2 \nabla^2\Psi }{a^2 }  - \frac{6\mathcal{H}\Psi'}{a^2 } \\
G^{0}_{\ i} &=& - G^i_{\ 0} = \frac{2\mathcal{H}  {\partial_i} \Phi   }{a^2 } + \frac{2  {\partial_{i}} \Psi' }{a^2 }  \\
G^{i}_{\ i} &=& {\gray - \frac{\mathcal{H}^2}{a^2} + \frac{2 a''}{a^3 } } \nonumber \\
&& + \left(\frac{2\mathcal{H}^2}{a^2}- \frac{4 a''}{a^3 } \right)\Phi - \frac{2\mathcal{H} \Phi'}{a^2} - \frac{2 \Psi''}{a^2 }- \frac{4\mathcal{H}\Psi'}{a^2 } \nonumber \\
&&  + \frac{(\partial_i^2-\nabla^2) (\Phi-\Psi) }{a^2 } ,\ \text{($i$ not summed)} \\
G^{i}_{\ j} &=&   \frac{ {\partial_i\partial_j} (\Phi-\Psi) }{a^2 } , \ (i\ne j)  \label{eq:shear}
 \end{eqnarray}
\end{frame}

  \begin{frame}
    \frametitle{傅立叶空间的 Einstein 张量 (重复指标求和)}
      {\blue
 \begin{eqnarray}
G^{0}_{\ 0} &=& {\gray \frac{3\mathcal{H}^2}{a^2 }} - \frac{6 \Phi  \mathcal{H}^2}{a^2 } - \frac{2k^2\Psi }{a^2 }  - \frac{6\mathcal{H}\Psi'}{a^2 } \\
\ii\hat{k}_i G^{i}_{0} &=&  \frac{2 k \mathcal{H} }{a^2 }\Phi + \frac{2 k \Psi' }{a^2 }  \\
G^{i}_{\ i} &=&  {\gray - \frac{3\mathcal{H}^2}{a^2} + \frac{6 {a''}}{a^3 } } \nonumber \\
&&  + \left(\frac{2\mathcal{H}^2}{a^2 }- \frac{4 {a''}}{a^3 } \right)\Phi - \frac{6\mathcal{H} \Phi'}{a^2 } - \frac{6 \Psi'' }{a^2 }- \frac{12\mathcal{H} \Psi'}{a^2 } \nonumber \\
&&  + \frac{2k^2 (\Phi-\Psi) }{a^2 } , \\
 \left(\hat{k}_i\hat{k}^j-\frac{1}{3}\delta^j_{\ i}\right)G^i_{\ j} &=&  \frac{2k^2(\Psi-\Phi)}{3a^2 } ,
 \end{eqnarray}}
  \end{frame}


  \begin{frame}
    \frametitle{小目标的checklist}
    \bitem
  \item[\checkmark]{给出度规微扰的数学描述(难度系数 $\star\star$)}
  \item[\checkmark]{计算爱因斯坦张量(难度系数 $\star\star\star\star$)}
  \item[3]{计算各种成分的能量动量张量(难度系数 $\star\star\star$)}    
  \item[4]{写出CDM的运动方程(难度系数 $\star$)}
  \item[5]{写出baryon的运动方程(难度系数 $\star\star$)}        
  \item[6]{写出massless neutrino的玻尔兹曼方程(难度系数 $\star\star\star\star\star$)}
  \item[7]{写出photon的玻尔兹曼方程(难度系数 $\star\star\star\star\star\star\star\star\star\star$)}
  \item[8]{把所有方程写成代码(难度系数$\star\star\star\star\star\star\star\star$)}
    \eitem
  \end{frame}
  
  \section{Energy-momentum Tensor}
  \secpage{微目标3}{能量动量张量}

   \begin{frame}
    \frametitle{暗物质和重子:理想流体近似}
    在我们感兴趣的红移范围,暗物质和重子的典型的热动能相比于它们的静止质量有亿点点低。 所以我们几乎可以忽略这些家伙在小范围内来回乱蹿的行为,只研究它们的集体(专业名词叫流体元)的行为。

    \bcenter
    \tfig{0.8}{chicken_slow.jpg}    \tfig{0.8}{chicken_fast.jpg}
    \ecenter

  \end{frame}


   \begin{frame}
     \frametitle{理想流体的能量动量}
     广义相对论中有个熟知的结论: 理想流体的能量动量张量是
     \tbox{$$T^\mu_{\ \nu} = (\rho+p)u^\mu u_\nu - pg^\mu_\nu$$}
     这里 $\rho, p$ 是和流体元共动的观测者看到的流体能量密度和压强,$u^\mu$ 是流体元的四维速度。
   \end{frame}


  \begin{frame}
    \frametitle{密度扰动 $\delta$}
    我们用 $\delta$ 表示理想流体的相对密度扰动,即

    {\blue $$\delta\equiv \frac{\rho - \bar{\rho}}{\bar{\rho}}$$}

    这里的 $\bar{\rho}$ 是“背景密度”(或理解为宇宙平均能量密度;CDM和重子的 $\bar{\rho}$ 都和 $a^{-3}$ 成正比)。$\rho$ 是和流体元共动的观测者测量到的流体能量密度。


    \notes{注意这里的 $\rho$ 是和流体元共动的观测者看到的密度;我们之前讨论冷物质的结构增长方程时定义的密度是相对于坐标系静止(即 $\vec{x}$ 不变)的参考者看到的密度。对冷物质而言,非相对论速度贡献的动能是二阶小量,因此两者在线性近似的意义上是相同的。}
  \end{frame}


  \begin{frame}
    \frametitle{坐标速度的归一化散度(简称速度散度)}
    我们把流体元的三维坐标速度记为 $\vec{u} = \frac{d\vec{x}}{d\tau}$ (在一阶近似意义下,这实际上是刨去宇宙膨胀后的“物理速度”)。

    坐标速度 $\vec{u}$ 在傅立叶空间的归一化散度记为
    {\blue $$\upsilon  \equiv i\hat{k} \cdot \vec{u}$$}
     因为要用 $\hat{k}$ 的缘故,$\upsilon$ 只能在傅立叶空间定义。

  \end{frame}


    \begin{frame}
    \frametitle{理想流体的 $T^\mu_{\ \nu}$}
    \bitem
    \item{流体的共动参考系密度  $\rho = \bar{\rho}(1+\delta)$;}
    \item{ 流体元四维速度的一阶近似为 $u^\mu = \frac{1}{a} (1-\Phi, \vec{u})$ 或者 等价地 $u_\mu = a(1+\Phi, -\vec{u})$。}
    \item{设流体的“背景状态方程”为 $w$,在共动参考系的声速平方为 $c_s^2 = \frac{p-\bar{p}}{\rho-\bar{\rho}}=\frac{p-w\bar{\rho}}{\delta \bar{\rho}}$。即共动参考系流体的压强 $p  = (w + c_s^2\delta)\bar{\rho}$。}
      \eitem
      于是根据流体的能量动量张量公式,有
     
  \end{frame}
  
  \begin{frame}
    \frametitle{理想流体的  $T^\mu_{\ \nu}$(续)}
    \begin{eqnarray}
      T^0_{\ 0} &=& {\gray \bar{\rho}} + \bar{\rho}\delta, \\
      T^0_{\ i} &=& - T^i_{\ 0} = -(1+w)\bar{\rho}u^i,  \\
      T^i_{\ i} &=&  {\gray -w\bar{\rho}} - c_s^2\bar{\rho}\delta  ,\ \text{($i$ not summed)} \\
      T^i_{\ j} &=& 0.\ (i\ne j) \label{eq:Tshear}
    \end{eqnarray}
    比较方程 \eqref{eq:shear} 和 \eqref{eq:Tshear},我们可以从爱因斯坦方程推出一个很有用的结论:{\blue 如果宇宙中仅含有理想流体,那么在线性微扰的近似下,牛顿引力势和曲率势相等 ($\Phi = \Psi$)。} 实际上,在CDM和重子主导的晚期宇宙,情况的确如此。(等等,宇宙学常数会搞破坏吗?其实并不会,因为宇宙学常数可以看成密度和压强严格均匀的无扰动理想流体。)
  \end{frame}


  \begin{frame}
    \frametitle{中微子对 anisotropic shear 的贡献}
    中微子这种可以长距离滑翔的家伙肯定就不能当成理想流体了。事实上,中微子的 $T^i_{\ j}$ ($i\ne j$) (它叫 {\blue anisotropic shear},中文也许可能大概兴许叫“各向异性剪切”……吧?) 确实明显地非零。

    \skipline

    
    在早期宇宙还有大量的自由电子的时候,光子和电子的碰撞足够频繁,光子的平均自由程足够短,就可以把一堆光子的杂乱无章的“内部无规则运动”忽略掉,把光子和重子当成耦合在一起的理想流体(photon-baryon fluid)。虽然在氢的recombination完成之后,光子也可以像中微子那样长途滑翔,不再是理想流体,但是这时候光子占宇宙总能量密度的比重已经不大了。所以光子对宇宙的 anisotropic shear的贡献始终不大。

      \skipline

      结论就是:{\blue 宇宙线性微扰的 anisotropic shear 的主要贡献来源于中微子。}
  \end{frame}


\begin{frame}
  \frametitle{傅立叶空间的理想流体能量动量张量}
  最后,我们把理想流体的能量动量张量都写到傅立叶空间(重复指标求和)
   {\blue \begin{eqnarray}
      T^0_{\ 0} &=& {\gray \bar{\rho}} + \bar{\rho}\delta, \label{eq:1}\\
      \ii \hat{k}_i T^i_{\ 0} &=&  (1+w)\bar{\rho}\upsilon,  \\
      T^i_{\ i} &=&  {\gray -3w\bar{\rho}} - 3c_s^2\bar{\rho}\delta, \\
      \left(\hat{k}_i\hat{k}^j - \frac{1}{3}g_i^j\right)T^i_{\ j} &=& 0. \label{eq:4}
    \end{eqnarray}
   }
    \skipline

    下面,我们要讨论无法用理想流体描述的中微子和光子——
\end{frame}


  \begin{frame}
    \frametitle{描述中微子的数学工具: 相空间分布函数}
    中微子要用相空间的分布函数 $f(\tau, x^1, x^2, x^3, p_1, p_2, p_3)$ 来描述。这里的 $p_\mu$ 是粒子四维动量的协变形式(这里只用到它的空间分量 $p_i$)。$f$ 的物理意义是六维的相空间 $(x^1,x^2,x^3, p_1, p_2, p_3)$内粒子数密度(因为它随 $\tau$ 变化,所以其实是七元函数)。也就是说:
      $$ f(\tau, x^1, x^2, x^3, p_1, p_2, p_3) dx^1dx^2dx^3dp_1dp_2dp_3$$
      代表了 $\tau$ 时刻在相空间体积元内的粒子数。

  \end{frame}



  \begin{frame}
    \frametitle{很好用的 $\vec{q}$ }
    我的PhD老板 J. Richard Bond 和他的合作者 Szalay 在上个世纪八十年代就发现,用坐标系静止观测者看到的物理动量和 $a$ 的乘积 $\vec{q}\equiv a \vec{p}_{\rm obs}$ 来描述分布函数是更为巧妙的做法,可以省去一大堆麻烦。这个其中的原因很微妙,可能需要你在后续的计算中逐步体会。

    \skipline

    为了避免混乱,我们始终只把 $\vec{q}$ 当成欧式空间的三维矢量,也就是说我们将遵循 {\blue $q_i = q^i$} 的规则,并只让 $\vec{q}$ 和三维欧式空间的矢量(常见的是 $\vec{k}$) 发生指标收缩(即取内积)。

    \skipline      

    我们把 $\vec{q}$ 的大小记作 $q$,并把归一化的 $\vec{q}$ 记作
    $$\hat{q} \equiv \frac{\vec{q}}{q}.$$
  \end{frame}

  \begin{frame}
    \frametitle{$\vec{q}$ 和 $p^\mu$ 的联系}
    虽然 $\vec{q}$ 在大多数情况下很好用,但是一旦涉及四维协变张量的概念(例如能量动量张量和测地线方程),我们还是要使用协变形式的 $p^\mu$。

    因此,我们需要建立 $\vec{q}$ 和 $p^\mu$ 之间的联系。这个必须用到广义相对论的观测标架理论,附录里一顿操作后给出{\blue
    \begin{eqnarray}
     p_i = -(1-\Psi) q_i,\ p^i = \frac{1}{a^2}(1+\Psi)q^i; \nonumber \\
     p_0 = (1+\Phi) \sqrt{q^2+m^2a^2},\ p^0 = \frac{1}{a^2}(1-\Phi)\sqrt{q^2+m^2a^2}. \label{eq:pq}
    \end{eqnarray}}
    
  \end{frame}


  
  \begin{frame}
    \frametitle{分布函数的零级近似}
    采用了变量 $\vec{q}$ 之后,分布函数就写成 $f(\tau, \vec{x}, \vec{q})$。

    \skiplines
    
    对于光子或中微子,我们可以写出相空间分布函数的零级近似
    $$ f_0 (q) = \equiv \frac{g_s}{(2\pi)^3}\frac{1}{e^{\sqrt{q^2+m^2a^2}/T_0} \pm 1} . $$
    对中微子 $T_0 = 1.95\SIK$;对光子  $T_0 = 2.726\SIK$, $m=0$。$g_s$ 是内禀自由度。
  \end{frame}


  \begin{frame}
    \frametitle{分布函数的微扰(从七元函数到四元函数)}
    然后我们要考虑坐标系共动观测者看到的分布函数 $f(\tau, \vec{x}, \vec{q})$ 对 $f_0(q)$ 的偏离。 对 $\vec{x}$ 坐标进行傅立叶变换之后,分布函数可以写成:
    $$ f(\tau,\vec{k}, \vec{q}) = f_0(q) + \delta f (\tau, \vec{k},\vec{q}) $$
    根据对称性,$\delta f$ 的线性演化只和 $\vec{q}$ 和 $\vec{k}$ 的夹角,以及矢量大小 $k$, $q$ 有关。所以其实我们要演化的是
    $$ \delta f(\tau, k, q, \mu) $$
    这里的 {\blue $\mu = \frac{\vec{k}\cdot\vec{q}}{kq}$ 是 $\vec{k}$ 和 $\vec{q}$ 夹角的余弦。}
    
    \notes{初始条件并非只和 $k, q, \mu$ 相关。但我们这里只是计算线性增长因子,所以并无影响。另外,初始条件最后也只是做统计平均处理,单独某个初始条件并不会出现在具体的计算中。}
  \end{frame}

  \begin{frame}
    \frametitle{极端相对论情况:从四元函数到三元函数}
    对极端相对论粒子,由于其能量动量张量的一阶微扰只涉及到如下形式的积分(这个我马上会紧接着展示给你看):
    $$\int \delta f (\tau, k, q, \mu) q^3 dq $$
    所以我们定义{\blue 对$q$积分后的分布函数微扰}:
    $$ F(\tau, k, \mu) \equiv \frac{\int \delta f (\tau, k, q, \mu) q^3 dq}{\int f_0(q) q^3 dq} $$
  \end{frame}


    \begin{frame}
      \frametitle{Legendre多项式展开:从三元函数到一堆二元函数}
      最后,我们把 $F(\tau, k, \mu)$ 进行Legendre多项式展开:
      $$ F(\tau, k, \mu) = \sum_{\ell=0}^\infty (-\ii)^\ell F_\ell(\tau, k) P_\ell(\mu) $$
      对固定的 $k$,我们要演化的就是一堆 $F_0(\tau, k), F_1(\tau, k),\ldots$。

      \skipline
      
      对忽略质量的中微子和忽略极化的光子而言,情况确实就是如此。但光子的极化会影响到光子和电子的散射截面大小,所以其实我们后面还要额外加一些描述极化的变量。
    \end{frame}

    \begin{frame}
      \frametitle{从分布函数到能量动量张量}
      按照广义相对论的粒子能量动量张量理论,在坐标 $\vec{x}_0$ 的单粒子的能量动量张量是
      $$ T^\mu_\nu = \frac{p^\mu p_\nu}{p^0\sqrt{-g}}\delta(\vec{x}-\vec{x}_0) $$
      所以一堆粒子的能量动量张量是
      \begin{equation}
        T^\mu_{\ \nu}(\tau, x^1, x^2, x^3) = \iiint \frac{p^\mu p_\nu}{p^0\sqrt{-g}}f(x^1, x^2, x^3, p_1, p_2, p_3) dp_1 dp_2dp_3. \label{eq:Tmunu}
      \end{equation}
  \end{frame}


    
    \begin{frame}
      \frametitle{能量动量张量}
      利用
      $$ \int f(\tau, k, q, \mu) q^3 dq =   \left[1+\sum_{\ell=0}^\infty(-\ii)^\ell F_\ell(\tau, k)P_\ell(\mu)\right] \int f_0(q) q^3 dq $$
      结合方程 \eqref{eq:Tmunu} 和 \eqref{eq:pq},并利用Legendre多项式的正交归一化公式 $\int_{-1}^1P_m(\mu)P_n(\mu) d\mu = \frac{2}{2m+1}\delta_{mn}$,可以算出傅立叶空间的能量动量张量(重复指标求和):
     {\blue \begin{eqnarray}
      T^0_{\ 0} &=& {\gray \bar{\rho}+}\bar{\rho} F_0 \label{eq:f1}\\
      \ii \hat{k}_i T^i_{\ 0} &=& \frac{1}{3}\bar{\rho} F_1 \\
      T^i_{\ i} &=& {\gray -\bar{\rho}} - \bar{\rho} F_0,  \\
     \left(\hat{k}_i\hat{k}^j-\frac{1}{3}g_i^j\right)T^i_{\ j} &=& \frac{2}{15}\bar{\rho}F_2. \label{eq:f4}
      \end{eqnarray}}
    \end{frame}
    


    \begin{frame}
      \frametitle{能量动量张量(续)}
      前面的
      $$ \bar{\rho} = \frac{4\pi}{a^4} \int q^3dq f_0(q) $$
      对比理想流体的方程 (\ref{eq:1}-\ref{eq:4}) 和极端相对论粒子气体的 (\ref{eq:f1}-\ref{eq:f4}),我们得到极端相对论粒子气体的等效流体描述。
      {\blue $$\delta = F_0,\ \upsilon= \frac{1}{4}F_1,\ w=c_s^2=\frac{1}{3}$$}
      但是由于存在高阶的 $F_\ell$ ($\ell\ge 2$) 修正,极端相对论粒子气体一般不能简单地当成理想流体。
    \end{frame}
    
    
  
  \begin{frame}
    \frametitle{小目标的checklist}
    \bitem
  \item[\checkmark]{给出度规微扰的数学描述(难度系数 $\star\star$)}
  \item[\checkmark]{计算爱因斯坦张量(难度系数 $\star\star\star\star$)}
  \item[\checkmark]{计算各种成分的能量动量张量(难度系数 $\star\star\star$)}    
  \item[4]{写出CDM的运动方程(难度系数 $\star$)}
  \item[5]{写出baryon的运动方程(难度系数 $\star\star$)}        
  \item[6]{写出massless neutrino的玻尔兹曼方程(难度系数 $\star\star\star\star\star$)}
  \item[7]{写出photon的玻尔兹曼方程(难度系数 $\star\star\star\star\star\star\star\star\star\star$)}
  \item[8]{把所有方程写成代码(难度系数$\star\star\star\star\star\star\star\star$)}
    \eitem

    \skipline
    
    好了别揪头发了,下一讲我们继续——
  \end{frame}


  \section{Appendix}
  \secpage{附录}{四维动量 $p^\mu$ 和 $\vec{q}$ 的关系}

  \begin{frame}
    \frametitle{观测者的局域Minkowski标架}
    根据广义相对论的观测者局域标架理论:一个坐标系静止观测者(即 $x^1, x^2, x^3$ 坐标固定的观测者)的四维速度是 $u^\mu = \frac{1}{a}(1-\Phi, 0, 0, 0)$。时间标架是
    $$e^\mu_0 = u^\mu = \frac{1}{a}(1-\Phi, 0, 0, 0).$$
    局域空间标架可以取为
    \bea
    e^{\mu}_1 &=& \frac{1}{a}(0, 1+\Psi, 0, 0) \newl
    e^{\mu}_2 &=& \frac{1}{a}(0, 0, 1+\Psi, 0) \newl
    e^{\mu}_3 &=& \frac{1}{a}(0, 0, 0, 1+\Psi) 
    \eea
  \end{frame}

  \begin{frame}  
  \frametitle{四维动量的空间分量和 $\vec{q}$ 的关系}
    坐标系静止观测者看到粒子的物理动量为
    $$ P^i = - P_i = - e^\mu_i p_\mu = -\frac{1+\Psi}{a} p_i $$
    按照定义 $P^i = \frac{q^i}{a} = \frac{q_i}{a}$, 就得到
    $$ q_i = -(1+\Psi)p_i$$
    或者一阶近似意义下等价地
    $$p_i = -(1-\Psi)q_i; q^i = \frac{1+\Psi}{a^2}q^i$$    
  \end{frame}

  \begin{frame}  
  \frametitle{四维动量的时间分量 和 $\vec{q}$ 的关系}
  坐标系观测到的物理能量
  $$P^0 = P_0 = e_0^\mu p_\mu = \frac{1-\Phi}{a} p_0$$
  按照静止质量的定义,它应该等于 $\sqrt{\frac{q^2}{a^2}+m^2}$,因此在一阶近似的意义下:
  $$p_0 = (1+\Phi)\sqrt{q^2+m^2a^2}, p^0 = \frac{1-\Phi}{a^2}\sqrt{q^2+m^2a^2}$$
  \end{frame}
  \ech
\end{document}




  
