\documentclass[CJK,13pt]{beamer}
\input{macros.tex}
\input{titlepage.tex}
  \date{}
  \begin{document}
  \bch
  \tpage{12}{Matter Power Spectrum and CMB Power Spectrum}

  \section{Matter Power Spectrum}
  \secpage{下面我们介绍如何用前面介绍的微扰计算结果来计算宇宙学观测量}{\addfig{1}{qidai.jpg}}

  
  \section{Matter Power Spectrum}

  \secpage{物质扰动功率谱}{$$P_m(k; z) = \frac{8\pi^2a^2k}{9H_0^2\Omega_m^2} \lvert\Phi(k;z)\rvert^2 \Delta^2(k).$$}
  
  \begin{frame}
    \frametitle{冷物质密度扰动}
    利用泊松方程,可以从牛顿引力势算出“冷物质密度扰动” $\delta_m$:
    $$ 4\pi G\bar{\rho}_m \delta_m = -\frac{k^2}{a^2}\Phi $$
    注意密度扰动是依赖于规范的量,这里的 $\delta_m$ 和牛顿规范下的 $\delta_c, \delta_b$ 并不完全一致。它们仅在晚期宇宙、远小于视界尺度($k\gg aH$) 时一致。在忽略辐射和中微子的情况下,$\delta_m$ 和物质共动同步规范下的冷物质密度扰动一致。
  \end{frame}

  \begin{frame}
    \frametitle{冷物质密度扰动功率谱}
    利用 $\bar{\rho}_m = \frac{3H_0^2}{8\pi G}\Omega_m a^{-3}$,可以写出冷物质密度扰动的功率谱为
    $$P_m(k; z) = |\delta_m(k)|^2 = \frac{4a^2k^4}{9H_0^2\Omega_m^2} \lvert\Phi(k;z)\rvert^2 P_\zeta(k)$$
    我们代码算出来的 $\Phi(k; z)$ 实际上是隐去了原初扰动初始随机变量 $\zeta(\vec{k})$ 的因子,所以上面补回了 $\zeta$ 的功率谱。

    用 $\zeta$ 的无量纲功率谱  $\Delta^2(k)=\frac{k^3}{2\pi^2}P_\zeta(k)$ 表示就是
    $$P_m(k; z) = \frac{8\pi^2a^2k}{9H_0^2\Omega_m^2} \lvert\Phi(k;z)\rvert^2 \Delta^2(k).$$
  \end{frame}


  \begin{frame}
    \frametitle{废话不多,直接上代码}
    \url{http://zhiqihuang.top/codes/matterpower.py}

    \skipline
    
    因为可能需要输出很大的 $k$ 的 $P_m(k)$,代码里对大的 $k$ 有特殊的处理方式:主要是把晚期宇宙的辐射和中微子的多极展开中的 $\ell>2$ 项进行人为压缩。物理上,由于晚期宇宙辐射和中微子占比低,且在小尺度($k\gg aH$)上很均匀,所以这个近似应该OK(数值上,你也可以通过改变压缩量来检测事实是否如此)。
  \end{frame}

  \section{CMB power spectrum}
  
  \secpage{CMB power spectrum}{加法公式又来了……}

  \begin{frame}
    \frametitle{宇宙微波背景辐射的光强扰动}
    由于再电离的光深不大,原初宇宙光子可以在地球附近被观测到。由于目前宇宙原初光子的温度 $T_{\rm CMB}=2.726\SIK$ 在微波波段,这些光子就被称为宇宙微波背景辐射(Cosmic Microwave Background),简称CMB。当然,实际的观测并非如此简单,我们至少需要避开银河系和河外星系的星光的干扰。如果是地面上的观测,我们还要考虑大气干扰等各种因素。

    \skipline

    我们可以在两个方向 $\vec{n}_1$ 和 $\vec{n}_2$ 观测光子的分布函数扰动(偏离黑体谱)。一般我们只关心对 $q^3dq$ 积分后的光强扰动。它可以由傅立叶逆变换给出:    
    $$\frac{\Delta I}{I}(\tau, \vec{x}, \vec{n}) = \frac{1}{(2\pi)^3}\int d^3\vec{k} \sum_{\ell}(-i)^\ell F_{\gamma\ell}(\tau, k)\zeta(\vec{k}) P_{\ell}(\vec{n}\cdot\hat{k})e^{i\vec{k}\cdot\vec{x}}$$

  \end{frame}

  \begin{frame}
    \frametitle{宇宙微波背景辐射温度扰动}    
    我们只能在 $\tau = \tau_0$ (红移零处)观测CMB,另外根据宇宙学原理,也可以不妨取地球附近 $\vec{x} = 0$ (这一步不是必须的,即使不嫌麻烦留着 $\vec{x}$, 最后也会消掉)。另外由于 $I\propto T^4$,所以观测到的
    $$\frac{\Delta T}{T}(\vec{n}) =\frac{1}{4} \frac{1}{(2\pi)^3}\int d^3\vec{k} \sum_{\ell}(-i)^\ell F_{\gamma\ell}(\tau_0, k)\zeta(\vec{k}) P_{\ell}(\vec{n}\cdot\hat{k}) $$
    由于存在随机变量 $\zeta(\vec{k})$,我们无法预言任何一个方向的 $\frac{\Delta T}{T}(\vec{n})$。所以,我们的兴趣转移到了两个不同方向的温度扰动的统计关联:
  \end{frame}

  \begin{frame}
    \frametitle{宇宙微波背景辐射的角关联函数}
    考虑 $\vec{n_1}$, $\vec{n_2}$ 两个方向上 $\frac{\Delta T}{T}$ 的统计关联(这个的意思是,按照暴胀理论,假想随机生成很多个宇宙并对这些宇宙中取平均;我们的宇宙只是随机生成的一个,所以实际观测到的量会偏离平均,这种偏离称为 cosmic variance):
    \bea
     && \langle \frac{\Delta T}{T}(\vec{n}_1)  \frac{\Delta T}{T}(\vec{n}_2)\rangle \newl
    &=& \frac{1}{16} \frac{1}{(2\pi)^6}\sum_{\ell,\ell'}(-i)^{\ell+\ell'} \times \newl
    && \int d^3\vec{k} \int d^3\vec{k}' F_{\gamma\ell}(\tau_0, k)F_{\gamma\ell'}(\tau_0, k')\langle \zeta(\vec{k})\zeta(\vec{k}')\rangle P_{\ell}(\vec{n_1}\cdot\hat{k}) P_{\ell}(\vec{n_2}\cdot\hat{k}')
    \eea
    
  \end{frame}

  \begin{frame}
    \frametitle{宇宙微波背景辐射的角关联函数(续)}
    代入原初曲率扰动功率谱的定义:
    $$ \langle \zeta(\vec{k})\zeta(\vec{k}')\rangle = (2\pi)^3\delta(\vec{k}+\vec{k}')\frac{2\pi^2}{k^3}\Delta^2(k)$$
      并对 $\vec{k}'$ 积分,得到
    \bea
     && \langle \frac{\Delta T}{T}(\vec{n}_1)  \frac{\Delta T}{T}(\vec{n}_2)\rangle \newl
    &=& \frac{1}{64 \pi}\sum_{\ell,\ell'}(-i)^{\ell+\ell'} \times \newl
    && \int d^3\vec{k}  F_{\gamma\ell}(\tau_0, k)F_{\gamma\ell'}(\tau_0, k)\frac{\Delta^2(k)}{k^3} P_{\ell}(\vec{n_1}\cdot\hat{k}) P_{\ell}(-\vec{n_2}\cdot\hat{k})
    \eea
  \end{frame}

  \begin{frame}
    \frametitle{宇宙微波背景辐射的角关联函数(续)}    
    回忆下球谐函数的加法公式:
    $$ P_\ell(\vec{n}\cdot\hat{k}) = \frac{4\pi}{2\ell+1}\sum_{m=-\ell}^\ell Y_{\ell m}^*(\vec{n})Y_{\ell m}(\hat{k}) .$$
    就有
    \bea
     && \langle \frac{\Delta T}{T}(\vec{n}_1)  \frac{\Delta T}{T}(\vec{n}_2)\rangle \newl
    &=&  \frac{1}{64 \pi}\sum_{\ell,\ell'}\frac{4\pi}{2\ell+1}\frac{4\pi}{2\ell'+1}\sum_{m, m'}(-i)^{\ell+\ell'}Y_{\ell m}^*(\vec{n_1})Y_{\ell' m'}(-\vec{n_2}) \times \newl
    && \int d^3\vec{k}  F_{\gamma\ell}(\tau_0, k)F_{\gamma\ell'}(\tau_0, k)\frac{\Delta^2(k)}{k^3}Y_{\ell m}(\hat{k}) Y_{\ell' m'}^*(\hat{k}) 
    \eea
  \end{frame}
  

    \begin{frame}
    \frametitle{宇宙微波背景辐射的角关联函数(续)}    
      把积分 $d^3\vec{k}$ 写成 $k^2dk d\Omega$,利用球谐函数的正交归一化性质,
      $$\int d^2\Omega Y_{\ell m}(\hat{k}) Y_{\ell' m'}^*(\hat{k}) = \delta_{\ell\ell'}\delta_{mm'}$$
      并对 $\ell', m'$ 求和, 就有
      \bea
      && \langle \frac{\Delta T}{T}(\vec{n}_1)  \frac{\Delta T}{T}(\vec{n}_2)\rangle \newl
      &=& \frac{1}{64 \pi}\sum_{\ell,m}\left[\frac{4\pi}{2\ell+1}\right]^2(-1)^\ell Y_{\ell m}^*(\vec{n_1})Y_{\ell m}(-\vec{n_2}) \times \newl
      && \int_0^\infty  \left[F_{\gamma\ell}(\tau_0, k)\right]^2\Delta^2(k)\frac{dk}{k}
    \eea
  \end{frame}


    \begin{frame}
    \frametitle{宇宙微波背景辐射的角关联函数(续)}    
      利用球谐函数的宇称性质 $(-1)^\ell Y_{\ell m}(-\vec{n}_2) =Y_{\ell m}(\vec{n}_2)$,就有
      \bea
      && \langle \frac{\Delta T}{T}(\vec{n}_1)  \frac{\Delta T}{T}(\vec{n}_2)\rangle \newl
      &=& \frac{1}{64 \pi} \sum_{\ell,m}\left[\frac{4\pi}{2\ell+1}\right]^2 Y_{\ell m}^*(\vec{n_1})Y_{\ell m}(\vec{n_2}) \times \newl
      && \int_0^\infty  \left[F_{\gamma\ell}(\tau_0, k)\right]^2\Delta^2(k)\frac{dk}{k}
    \eea
  \end{frame}
    

    \begin{frame}
    \frametitle{宇宙微波背景辐射的角关联函数(续)}    
      最后,对$\vec{n}_1, \vec{n}_2$ 利用球谐函数加法公式,得到
      \bea
      && \langle \frac{\Delta T}{T}(\vec{n}_1)  \frac{\Delta T}{T}(\vec{n}_2)\rangle \newl
      &=& \frac{1}{16} \sum_{\ell} \frac{1}{2\ell+1} P_\ell(\vec{n}_1\cdot\vec{n}_2) \int_0^\infty  \left[F_{\gamma\ell}(\tau_0, k)\right]^2\Delta^2(k)\frac{dk}{k}
    \eea
  \end{frame}


    \begin{frame}
    \frametitle{宇宙微波背景辐射的角关联函数(完结)}    
      定义CMB功率谱
      $$C_\ell = \frac{\pi}{4(2\ell+1)^2} \int_0^\infty  \left[F_{\gamma\ell}(\tau_0, k)\right]^2\Delta^2(k)\frac{dk}{k}.$$
      就有
      $$ \langle \frac{\Delta T}{T}(\vec{n}_1)  \frac{\Delta T}{T}(\vec{n}_2)\rangle = \sum_{\ell}\frac{2\ell+1}{4\pi}C_\ell P_\ell(\vec{n}_1\cdot\vec{n}_2)$$
  \end{frame}
    


    \begin{frame}
    \frametitle{Line of sight Integral}    
    由于实验可以观测到 $\ell\sim 10^4$ 的CMB功率谱。而我们在解Boltzmann方程时一般会在 $\ell_{\max}$ 约为十几左右做个截断,所以前面推导的公式只在很小的 $\ell$ 具有实际意义。

    \skipline

    实际的计算使用的是 $F_{\gamma\ell}$ 的一个积分表示:
    $$\frac{F_{\gamma\ell}(\tau_0, k)}{2\ell+1} = \int_0^{\tau_0} d\tau e^{-\kappa} S(k,\tau) j_\ell\left(k(\tau_0-\tau)\right)$$
    这里的 $\kappa$ 是积分得到的$\tau$ 和 $\tau_0$ 之间的总光深。源函数 $S(k,\tau)$ 可以用Boltzmann方程给出的 $\ell\le 2$ 的微扰量组合得出(请参考神作 astro-ph/9603033)。
    由于球贝塞尔函数 $j_\ell$ 都可以事先算好或直接用近似算法快速计算,所以这种计算方法可以实现秒算CMB功率谱。
  \end{frame}
    


    \begin{frame}
      \frametitle{回到CAMB和CLASS}
      最后,当你理解了所有计算量的来龙去脉之后,我希望你能回到CAMB或者CLASS而不是依赖于我写的示范代码 —— 毕竟,python太慢,而且中微子是有质量的。

      \skipline

      安装CAMB很简单:

      pip3 install camb

      然后你搜索下camb的示范代码,很快就能学会如何计算各种微扰量、物质功率谱和CMB功率谱。
    \end{frame}
    
    
    \ech
\end{document}




  
